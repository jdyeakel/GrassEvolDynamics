\documentclass{article}[10pt]

%\usepackage[margin=1in,footskip=0.25in]{geometry}

%\usepackage{helvet}
%\renewcommand{\familydefault}{\sfdefault}
\renewcommand\refname{\vskip -1cm}
%\renewcommand{\rmdefault}{phv} % Arial
%\renewcommand{\sfdefault}{phv} % Arial
\usepackage{setspace}
\usepackage{wrapfig}
\usepackage{amsmath}
\usepackage{amssymb}
\usepackage{graphicx}
\usepackage{mathrsfs}
\usepackage{bm}
\usepackage{wasysym}
\usepackage{placeins}
\usepackage{multirow}
\usepackage[T1]{fontenc}
\usepackage[comma]{natbib}
\usepackage{framed}
\usepackage{caption}
\usepackage{longtable}
\usepackage{geometry}
\usepackage{lineno}

\geometry{verbose,letterpaper,tmargin=2.54cm,bmargin=2.54cm,lmargin=2.54cm,rmargin=2.54cm}


\title{Proposal Summary: African food webs over space and time: using ecological and environmetnal constraints to predict community structure and function throughout the late Cenozoic}

\begin{document}

\maketitle

{\bf Background} Mammalian communities in Africa have undergone dramatic changes over the last five million years, and much of this change has been linked to the spread of C4 grassland habitats.
The \emph{Coupled Grassland and Mammalian Community Dynamics Over Ecological and Evolutionary Timescales} working group at Dartmouth College in Hanover, New Hampshire is the second meeting in a series with an aim to include researchers spanning multiple fields with an interest in these systems to generate original research.
The first working group was hosted and funded by the Santa Fe Institute in Santa Fe, New Mexico from September 20-24, 2015.
This follow-up working group meeting was hosted and funded by the Neukom Institute and Dartmouth College from May 22-23, 2016.
The participants included

\begin{enumerate}
\item Nathaniel Dominy (Dartmouth College)
\item Vivek Venkataraman (Dartmouth College)
\item Michael Brown (Dartmouth College)
\item Caley Johnson (CUNY)
\item Andy Dobson (Princeton University)
\item John Fryxell (University of Guelph)
\item Kevin Uno (Lamont-Doherty Earth Observatory)
\item Mathias Pires (S\~ao Paulo University via Skype)
\item Caroline Str\"omberg (University of Washington via Skype)
\item Justin Yeakel (University of California, Merced; Santa Fe Institute)
\end{enumerate}

\vspace{5 mm}

{\bf Products of this working group} Members of the working group have decided to initiate a large NSF proposal (NSF Macrosystems Ecology; \$2.6 Million), which is due October 27, 2016.
Our discussions focused on establishing an integrative theory to \emph{i}) predict the occurance of herbivore species in African habitats with known environmental constraints, and \emph{ii}) predict the interactions between species in habitats with known community compositions.
These complimentary approaches will use extensive modern and paleontological datasets to establish statistical models of herbivore occupancy as well as plant/herbivore and herbivore/carnivore interactions.
Model predictions will be verified based on known characteristics of both modern and paleontological African systems (for example the distribution of mammalian carbon isotope ratios within a given habitat, which describes the flow of different nutrient sourcepools throughout the food web).
Together, these \emph{trophic niche models} will be used to reconstruct community food webs over multiple trophic levels such that the system-level impact of changing environments, the evolution of novel species, as well as the extinction of species or functional groups of species, can be directly assessed.
If our models prove to have predictive accuracy in modern and paleontological ecosystems, it will then the next phase of the project will aim to make specific predictions of future food web composition and functioning based on future climate predictions in Africa.

A second emphasis of this proposal will be to assess the impact of hominin evolution and diversification African food webs over evolutionary time.
Of particular interest is how increasing dietary generalism and omnivory among hominins (and \emph{Homo} in particular) may have influenced the dynamic functioning of the food web.
Because we have a record of species extinctions over time, a central question of this project focus will involve determining whether the foraging traits associated with derived hominins may have increased the likelihood of species-specific extinctions.

\vspace{5 mm}

{\bf Primary Questions} The proposal will thus be designed to address the following larger-scope questions:

\begin{enumerate}
	\item Can trophic niche models predict modern assemblages across Africa?
	\item Can trophic niche models predict paleo assemblages spanning the last 5 million years?
	\item Can trophic niche models predict changes over time in any single assemblage?
	\item Can trophic niche models predict changes in hominin impacts on community food webs?
\end{enumerate}

\vspace{5 mm}

{\bf Future Directions} Our primary goal over the next few months is to construct and submit the above proposal, which is due on October 27, 2016.
As we begin to compile and compliment existing datasets to inform the development of the model, project co-PIs will aim to schedule a third follow-up meeting during late spring/summer 2017 at the Santa Fe Institute in Santa Fe, New Mexico, pending approval by the SFI Science Board.
This third meeting will be used to develop the roadmap for future efforts, refine the development of the model, as well as to identify the publication plan.
In addition to the NSF MacroSystems submission, and because a large portion of this project deals with the reconstruction and analysis of complex systems, we believe that a parallel effort to develop a McDonnell grant, and/or an NSF Sedimentary Geology and Paleobiology grant may be fruitful, particularly if the MacroSystems grant is not successfull.
Importantly, many elements of the proposed project can be tackeled with medium-smaller sized grants in the case that no large grants are successful.





\end{document}
